\documentclass[a4paper,12pt]{article}
\usepackage[utf8]{inputenc}
\usepackage{geometry}
\geometry{a4paper, margin=1in}
\usepackage{hyperref}
\usepackage{titlesec}
\usepackage{setspace}
\usepackage{lipsum} % Per testo segnaposto


\title{L'Operazione "Cronos" e l'impatto sul Diritto Informatico}
\author{Sohail Mama, Giovanni Donati}
\date{a.a 2024/2025}

\begin{document}

\maketitle

\vfill
\begin{center}
\textit{L'Operazione "Cronos" è stata una delle più significative azioni internazionali contro il cybercrime, mirata a smantellare il gruppo ransomware LockBit. Questo research work analizza il contesto giuridico dell'operazione e le implicazioni legali nel contrasto ai ransomware.    }
\end{center}
\vfill


 


\newpage
\tableofcontents
\newpage

\section{LockBit: Storia e Modus Operandi}
Lockbit è una cyber-gang russa, attiva da circa 4 anni. Il loro obiettivo è esclusivamente quello di fare soldi, tanto che quando iniziò la guerra Russia-Ucraina, si tirarono fuori da ogni discorso politico e di appoggio a qualsiasi fazione.
\newline
\newline
La gang opera con il metodo «ransomware as a service», ovvero si appoggia a piattaforme ransomware da cui acquisisce servizi di intrusione via via più sofisticati a seconda della piattaforma da violare. A comporre Lockbit sono circa 100 persone che dividono i compiti. C’è chi sviluppa il malware, c’è chi trova i punti di accesso e c’è chi invece effettua l’attacco utilizzando il malware sviluppato dai primi.

\section{L'Operazione Cronos}
\subsection{Il coordinamento internazionale}
Forze dell'ordine di 11 paesi hanno partecipato, coordinandosi per colpire l'infrastruttura del gruppo, sequestrare server e congelare fondi.

\subsection{Le misure legali adottate}
Arresti, sequestri e sanzioni sono stati gli strumenti principali per contrastare LockBit. L'operazione ha sollevato questioni giuridiche su giurisdizione e cooperazione internazionale.

\section{Implicazioni Giuridiche}
\subsection{Diritto informatico e criminalità digitale}
L'uso del ransomware solleva interrogativi su come inquadrare penalmente questi crimini e sulle strategie per il rafforzamento della cybersecurity.

\subsection{Ruolo della cooperazione internazionale}
L'Interpol, Europol e altre agenzie hanno evidenziato l'importanza di una risposta coordinata contro il cybercrimine.

\section{Conclusioni}
L'Operazione "Cronos" ha rappresentato un successo nella lotta ai ransomware, ma il fenomeno rimane una sfida aperta. È necessario rafforzare la cooperazione internazionale e le normative in materia di cybersecurity per prevenire nuovi attacchi.

\newpage
\section{Fonti:}
\hyperlink{https://www.corriere.it/tecnologia/23_dicembre_19/chi-sono-i-russi-di-lockbit-hacker-che-hanno-bloccato-la-pubblica-amministrazione-politica-contano-solo-i-soldi-695b2492-8908-4e12-993c-5cb23941dxlk.shtml}
{corriere}

\end{document}
