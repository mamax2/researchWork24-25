\documentclass[a4paper,12pt]{article}
\usepackage[utf8]{inputenc}
\usepackage{geometry}
\geometry{a4paper, margin=1in}
\usepackage{hyperref}
\usepackage{titlesec}
\usepackage{setspace}
\usepackage{lipsum} % Per testo segnaposto

\title{L'Operazione "Cronos" e l'impatto sul Diritto Informatico}
\author{Sohail Mama, Giovanni Donati}
\date{a.a 2024/2025}

\begin{document}

\maketitle

\vfill
\begin{center}
\textit{L'Operazione "Cronos" è stata una delle più significative azioni internazionali contro il cybercrime, mirata a smantellare il gruppo ransomware LockBit. Questo research work analizza il contesto giuridico dell'operazione e le implicazioni legali nel contrasto ai ransomware.}
\end{center}
\vfill

\newpage
\tableofcontents
\newpage

\section{LockBit: Storia e Modus Operandi}
Lockbit è una cyber-gang russa, attiva da circa 4 anni. Il loro obiettivo è esclusivamente quello di fare soldi, tanto che quando iniziò la guerra Russia-Ucraina, si tirarono fuori da ogni discorso politico e di appoggio a qualsiasi fazione.

La gang opera con il metodo «ransomware as a service», ovvero si appoggia a piattaforme ransomware da cui acquisisce servizi di intrusione via via più sofisticati a seconda della piattaforma da violare. A comporre Lockbit sono circa 100 persone che dividono i compiti. C’è chi sviluppa il malware, c’è chi trova i punti di accesso e c’è chi invece effettua l’attacco utilizzando il malware sviluppato dai primi.

\subsection{Ransomware: Origini e Sviluppi Storici}
Il primo attacco ransomware documentato risale al 1989 con il cosiddetto “AIDS Trojan”, un malware distribuito via floppy disk che cifrava i file degli utenti chiedendo il pagamento di un riscatto tramite posta. Da allora, l'evoluzione tecnologica e la diffusione globale di Internet hanno reso questi attacchi sempre più frequenti, sofisticati e dannosi. A partire dal 2017, con l'attacco WannaCry, il ransomware è diventato un problema sistemico a livello mondiale, con danni stimati in miliardi di dollari ogni anno.

\subsection{LockBit e il Dark Web}
LockBit ha fatto largo uso del dark web sia per la comunicazione interna sia per la pubblicazione dei dati rubati. Il loro portale, accessibile tramite rete Tor, includeva sezioni dedicate alle “vittime pubbliche”, con timer per la pubblicazione dei dati sensibili in caso di mancato pagamento. Questo uso strategico del dark web rappresenta una componente chiave della loro operatività, amplificando la pressione psicologica sulle vittime e aumentando la loro visibilità nel panorama criminale.

\section{L'Operazione Cronos}
\subsection{Il coordinamento internazionale}
Forze dell'ordine di 11 paesi hanno partecipato, coordinandosi per colpire l'infrastruttura del gruppo, sequestrare server e congelare fondi.

\subsection{Le misure legali adottate}
Arresti, sequestri e sanzioni sono stati gli strumenti principali per contrastare LockBit. L'operazione ha sollevato questioni giuridiche su giurisdizione e cooperazione internazionale.

\section{Implicazioni Giuridiche}
\subsection{Diritto informatico e criminalità digitale}
L'uso del ransomware solleva interrogativi su come inquadrare penalmente questi crimini e sulle strategie per il rafforzamento della cybersecurity.

\subsection{Ruolo della cooperazione internazionale}
L'Interpol, Europol e altre agenzie hanno evidenziato l'importanza di una risposta coordinata contro il cybercrimine.

\subsection{Esempi di Operazioni Simili}
Prima di Cronos, altre operazioni hanno segnato momenti cruciali nella lotta contro il cybercrime:
\begin{itemize}
    \item \textbf{Operazione Disruptor} (2020): Coordinata da Europol, ha portato alla chiusura di numerosi marketplace illegali sul dark web.
    \item \textbf{Operazione Tovar} (2014): Ha smantellato la rete dietro il malware GameOver Zeus e il ransomware CryptoLocker.
    \item \textbf{Emotet Takedown} (2021): Azione globale che ha disattivato una delle infrastrutture di malware più pericolose del decennio.
\end{itemize}
Questi esempi evidenziano come la collaborazione sovranazionale sia ormai una componente imprescindibile nella cybersicurezza.

\section{Rischi Futuri e Raccomandazioni Politiche}
\subsection{Tendenze future del ransomware}
Gli esperti prevedono un’evoluzione dei ransomware verso attacchi più mirati e persistenti, con tecniche di social engineering sempre più sofisticate. Inoltre, si teme l’uso di intelligenza artificiale da parte degli attori malevoli per automatizzare la ricerca di vulnerabilità nei sistemi.

\subsection{Raccomandazioni politiche e legislative}
Per far fronte a queste minacce emergenti, si propongono alcune misure:
\begin{itemize}
    \item Rafforzare la legislazione nazionale e armonizzare i reati informatici a livello UE.
    \item Istituire un’agenzia europea permanente contro il cybercrime.
    \item Promuovere partenariati pubblico-privato per lo scambio tempestivo di informazioni.
    \item Incentivare la formazione continua per i professionisti della sicurezza informatica.
\end{itemize}
Solo attraverso un impegno collettivo e duraturo sarà possibile affrontare con efficacia le nuove sfide digitali.

\section{Conclusioni}
L'Operazione "Cronos" ha rappresentato un successo nella lotta ai ransomware, ma il fenomeno rimane una sfida aperta. È necessario rafforzare la cooperazione internazionale e le normative in materia di cybersecurity per prevenire nuovi attacchi.

\newpage
\section{Fonti:}
\begin{itemize}
    \item \href{https://www.corriere.it/tecnologia/23_dicembre_19/chi-sono-i-russi-di-lockbit-hacker-che-hanno-bloccato-la-pubblica-amministrazione-politica-contano-solo-i-soldi-695b2492-8908-4e12-993c-5cb23941dxlk.shtml}{Corriere della Sera}
    \item \href{https://www.europol.europa.eu/media-press/newsroom/news/lockbit-ransomware-group-disrupted-in-international-cyber-police-operation}{Europol: Operazione Cronos}
    \item \href{https://www.bbc.com/news/technology-68322223}{BBC Technology}
    \item \href{https://www.interpol.int/en/News-and-Events/News/2024/Global-Operation-Cronos-strikes-Lockbit}{Interpol}
\end{itemize}

\end{document}
